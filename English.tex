\documentclass[a4paper, twoside]{report}
\usepackage[margin=0.5in]{geometry}
\usepackage{amsmath}
\usepackage{amsfonts}
\usepackage{amssymb}
\usepackage{amsthm}
\usepackage{graphicx}
\usepackage{physics}
\usepackage{tikz}
\usepackage{mathrsfs}
\usepackage{pgfplots}
\usepgfplotslibrary{polar}
\usepgfplotslibrary{fillbetween}
\pgfplotsset{compat=1.18}
\usetikzlibrary{arrows,patterns, backgrounds, calc,fadings,shadows.blur, shapes}
\theoremstyle{plain}
\usepackage[most]{tcolorbox}
\usepackage{physunits}
\usepackage{float}
\usepackage[stable]{footmisc}
\usepackage{xcolor}
\usepackage{wrapfig}
\usepackage{microtype}
\usepackage{thmtools}
\usepackage[framemethod=TikZ]{mdframed}
\usepackage{steinmetz}
\usepackage{adjustbox}
\usepackage{listings}
\usepackage[version=4]{mhchem}
\newcommand{\floor}[1]{\lfloor #1 \rfloor}
\newcommand*{\Perm}[2]{{}^{#1}\!P_{#2}}%
\newcommand*{\Comb}[2]{{}^{#1}C_{#2}}%
\mdfsetup{skipabove=1em,skipbelow=0em}
\tcbuselibrary{xparse}
\usepackage[font=small,labelfont=bf,margin=\parindent,tableposition=top]{caption}
\newenvironment{solution}
{\renewcommand\qedsymbol{$\blacksquare$}\begin{proof}[Solution]}
	{\end{proof}}


\declaretheoremstyle[
headfont=\bfseries\sffamily\color{blue!70!black}, bodyfont=\normalfont,
mdframed={
	linewidth=2pt,
	rightline=false, topline=false, bottomline=false,
	linecolor=blue, backgroundcolor=blue!5,
}
]{thmbluebox}
\declaretheorem[style=thmbluebox, numbered=no, name=Example]{eg}


\declaretheoremstyle[
headfont=\bfseries\sffamily\color{blue!70!black}, bodyfont=\normalfont,
numbered=no,
mdframed={
	linewidth=2pt,
	rightline=false, topline=false, bottomline=false,
	linecolor=blue, backgroundcolor=blue!1,
},
]{thmexplanationbox}
\declaretheorem[style=thmexplanationbox, name=Solution]{tmpexplanation}
\newenvironment{explanation}[1][]{\vspace{-10pt}\begin{tmpexplanation}}{\end{tmpexplanation}}


\declaretheoremstyle[
headfont=\bfseries\sffamily\color{red!70!black}, bodyfont=\normalfont,
mdframed={
	linewidth=2pt,
	rightline=false, topline=false, bottomline=false,
	linecolor=red, backgroundcolor=red!5,
}
]{thmredbox}
\declaretheorem[style=thmredbox, name=Theorem]{theorem}

\declaretheoremstyle[
headfont=\bfseries\sffamily\color{red!70!black}, bodyfont=\normalfont,
numbered=no,
mdframed={
	linewidth=2pt,
	rightline=false, topline=false, bottomline=false,
	linecolor=red, backgroundcolor=red!2,
},
qed=\qedsymbol
]{thmproofbox}
\declaretheorem[style=thmproofbox, name=Proof]{replacementproof}
\renewenvironment{proof}[1][\proofname]{\vspace{-10pt}\begin{replacementproof}}{\end{replacementproof}}

\declaretheoremstyle[
headfont=\bfseries\sffamily\color{brown!70!black}, bodyfont=\normalfont,
mdframed={
	linewidth=2pt,
	rightline=false, topline=false, bottomline=false,
	linecolor=brown, backgroundcolor=brown!5,
}
]{thmbrownbox}
\declaretheorem[style=thmbrownbox, numbered=no, name=Question]{asign}

\declaretheoremstyle[
headfont=\bfseries\sffamily\color{brown!70!black}, bodyfont=\normalfont,
numbered=no,
mdframed={
	linewidth=2pt,
	rightline=false, topline=false, bottomline=false,
	linecolor=brown, backgroundcolor=brown!1,
},
]{thmansbox}
\declaretheorem[style=thmansbox, name=Answer]{thmbrown}
\newenvironment{anse}[1][]{\vspace{-10pt}\begin{thmbrown}}{\end{thmbrown}}

\tikzset {_uty0p60aj/.code = {\pgfsetadditionalshadetransform{ \pgftransformshift{\pgfpoint{0 bp } { 0 bp }  }  \pgftransformrotate{-270 }  \pgftransformscale{2 }  }}}
\pgfdeclarehorizontalshading{_zo7gli6ny}{150bp}{rgb(0bp)=(0.6,0.85,1);
rgb(53.66071428571429bp)=(0.6,0.85,1);
rgb(61.60714285714286bp)=(0,0.5,0.5);
rgb(100bp)=(0,0.5,0.5)}
\tikzset{every picture/.style={line width=0.75pt}} %set default line width to 0.75pt 

\usepackage[hidelinks]{hyperref}

\begin{document}
	\begin{titlepage}
		\begin{center}
			\Huge{English (FCHS0105)}\\
			Srijan Mahajan (2023UCM2326)\\
			Prof. Komal Yadav (8700724830)
		\end{center}
	\end{titlepage}
	\tableofcontents
	\newpage
	\chapter{Vocabulary Enhancement}
	\section{Dictionary}
	A dictionary includes:
	\begin{itemize}
		\item Entry word - The word we are looking for
		\item Guide Words - The words at the top of the page. Indicate the first and last word on any page
		\item Syllables - The number of "sounds" a word is divided into
		\item Part of speech - If the word is a noun, adjective, etc.
		\item Pronunciation - By Phonetic symbols
		\item Meaning
		\item Inflected forms- Other forms like degree
		\item Synonyms/Antonyms
		\item Homophones\footnote{Similar sounding}/Homonyms\footnote{Similar spelling}
	\end{itemize}
	\section{Collocations}
	These are sets of words commonly used together. These are of the following types:
	\begin{itemize}
		\item Adverb \& Adjective
		\item Adjective \& Noun
		\item Verb \& Noun
		\item Verb \& Preposition
		\item Adjective \& Preposition
		\item Noun \& Verb
		\item Verb \& Noun \& Preposition
	\end{itemize}

	\begin{asign}[19/08/2023]
		Take up any monophthong or diphthong or consonant sound. Think of at least 10 words with that sound apart from the examples given in the app. Write a story putting best of your imagination skills using the 10 words you’ve found.
	\end{asign}
	The consonant sound: $/k/$. Ten words are:
	\begin{itemize}
		\item cat
		\item kite
		\item cookie
		\item kayak
		\item camera
		\item kangaroo
		\item curious
		\item captured
		\item snacks
		\item picnic
	\end{itemize}
	\begin{anse} 
		
		On a sunny day, I packed my backpack with \underline{cookies}, a \underline{kite}, and my \underline{camera} for a lakeside \underline{picnic}. As I set up my blanket, a \underline{curious} \underline{cat} approached, lured by the treats' aroma.
		
		After enjoying my \underline{snacks}, I rented a \underline{kayak} to explore the calm lake waters. The gentle breeze and serene surroundings made for a perfect escape. I even spotted a \underline{kangaroo} hopping along the shoreline, a charming surprise.
		
		As the sun dipped, I packed up, grateful for the memorable day. \underline{Captured} moments and tranquil kayak rides made it an adventure to cherish.
	\end{anse}
	\section{Subject Verb Agreement}
	It refers to the form of the verb used with noun/pronoun.\\
	\begin{itemize}
		\item With a $3^\text{rd}$ person subject, we place 's" at the end of the verb.
		\item Error of Proximity- If in a sentence, we have more than one subject then the verb is made to agree with the subject placed nearer to it in the sentence.
		\item If in a sentence we have two subjects joined with 'and', then a plural verb is required. E.g. Gold and silver \underline{are} precious metals. \textbf{Exception-} If subject pronouns/nouns suggest a single idea or refer to a single person/thing, the verb will be singular. E.g. Time and tide waits for none.
		\item If two nouns are qualified with 'each'/'every' even though they are connected by 'and' require a singular verb.
		\item If the two subjects are joined by 'with', 'as well as', etc. then the verb is singular. E.g. The ship, with the crew, was lost. Here, "with the crew" is a parenthetical phrase.
		\item If the two subjects are joined by 'either/or', 'neither/nor', the verb is singular.
		\item The words 'pain' and 'means' may take either singular or plural forms in a sentence but the sentence should be consistent in construction.
		\item The word 'none' is singular but takes plural verbs.
		\item A collective noun will always take a singular verb. \textbf{Exception-} If in the sentence, we refer to the individuals composing the noun, the verb used is plural.
		\item If in a sentence, we have a plural noun which is a proper name for some single object/collective unit, singular verb will be used.
		\item When the plural noun denoted some specific quantity/ amount considered as a whole, the verb used is singular, i.e. all units of measurement are accompanied with singular verbs.
	\end{itemize}
	\section{Redundancy}
	The following are the ways to reduce redundancy-
	\begin{itemize}
		\item Eliminate unnecessary adverbs.
		\item Eliminate/replace meaningless adjectives.
		\item Trimming long phrases.
	\end{itemize}
	\section{Question Tags}
	\begin{itemize}
		\item If the main clause is affirmative, the question is negative and vice-versa.
		\item If the main clause has an auxiliary verb, the question tag will also carry the auxiliary verb.
		\item If there is no auxiliary verb in  the main clause, then we use 'do', 'does', 'did' in the question tag.
		\item Negatives are usually contracted in the question tags, but this is not the case in formal speech.
		\item After imperative sentences, 'won't you' is used. E.g. Shut up, \underline{can you?}.
		\item Main clauses containing negative adverbs (never, seldom, hardly), carry a positive question tag.
		\item If the main clause contains a modal verb (should, could, can), the question tag will carry the same modal verb. 
	\end{itemize}
	\section{Reported Speech}
	\begin{itemize}
		\item When the reporting/principal verb is in the past tense, all the present tenses of the direct speech are changed into the corresponding past tenses.
		\item The tense may not change if the statement is still relevant in the indirect speech or is a universal truth.
		\item The pronouns of the direct speech are changed so that their relation with the reporter and the listener, rather than the original speaker is indicated.
		\item Words expressing proximity, in time or place are generally changed into the words expressing distance. \textbf{Exception- } The changes do not occur if the speech is reported during the same period or at the same place.
		\item In case of questioning in direct speech, we use words like 'inquired', 'asked', etc. while reporting it.
		\item In case of commands and requests in direct speech, we use words like 'ordered', 'requested', etc. while reporting it.
		\item In case of exclamations and wishes in direct speech, we use words like 'exclaimed', 'wished', etc. while reporting it.
	\end{itemize}
\end{document}