\section{Tutorial-2}
\subsection{Solve the following}
\begin{asign}
	Differentiate $x^{\sin x}$ w.r.t. $(\sin x)^x$.
\end{asign}
\begin{anse}
	Let,
	\[\begin{split}
		u&=x^{\sin x}\\
		\ln u&=\sin x \ln x\\
		\implies \dv{u}{x}&=x^{\sin x}\left(\cos x\ln x+\frac{\sin x}{x}\right)
	\end{split}
	\quad \bigg\vert
	\begin{split}
		v&=(\sin x)^x\\
		\implies \dv{v}{x}&=(\sin x)^x\left(\ln\sin x+x\cot x\right)
	\end{split}
	\]
	Thus,
	\[\begin{split}
		\dv{y}{x}&=\frac{\dv{u}{x}}{\dv{v}{x}}\\
		\therefore \dv{y}{x}&=\frac{x^{\sin x}\left(\cos x\ln x+\frac{\sin x}{x}\right)}{(\sin x)^x\left(\ln\sin x+x\cot x\right)}
	\end{split}\]
\end{anse}
\begin{asign}
	Differentiate $\ln(x)^{\tan x}$ w.r.t. $\sin(m\cos^{-1}x)$.
\end{asign}
\begin{anse}
	Let,
	\[\begin{split}
		u&=\ln(x)^{\tan x}\\
		\implies \dv{u}{x}&=(\ln x)^{\tan x}\left[\frac{\tan x}{x\ln x}+\sec^2x\ln(\ln x)\right]
	\end{split}
	\quad \bigg\vert
	\begin{split}
		v&=\sin(m\cos^{-1}x)\\
		\implies \dv{v}{x}&=-\frac{m\cos(m\cos^{-1}x)}{\sqrt{1-x^2}}
	\end{split}\]
	Thus,
	\[\begin{split}
		\dv{y}{x}&=\frac{\dv{u}{x}}{\dv{v}{x}}\\
		\therefore \dv{y}{x}&=-\frac{\sqrt{1-x^2}(\ln x)^{\tan x}\left[\frac{\tan x}{x\ln x}+\sec^2x\ln(\ln x)\right]}{m\cos(m\cos^{-1}x)}
	\end{split}\]
\end{anse}
\begin{asign}
	If $x^y=e^{x-y}$, then prove that $\dv{y}{x}=\frac{\ln x}{(1+\ln x)^2}$.
\end{asign}
\begin{anse}
	\[\begin{split}
		x^y&=e^{x-y}\\
		y&=\frac{x}{\ln x+1}\\
		\implies \dv{y}{x}&=\frac{\ln x}{(1+\ln x)^2}
	\end{split}\]
\end{anse}
\begin{asign}
	If $y=x^{x^x\cdots\infty}$, then show that $x\dv{y}{x}=\frac{y^2}{1-y\ln x}$.
\end{asign}
\begin{anse}
	\[\begin{split}
		y&=x^y\\
		\ln y&=y\ln x\\
		\implies \frac{1}{y}\dv{y}{x}&=\ln x\dv{y}{x}+\frac{y}{x}\\
		\therefore x\dv{y}{x}&=\frac{y^2}{1-y\ln x}
	\end{split}\]
\end{anse}
\begin{asign}
	If $y=\sqrt{\cos x+\sqrt{\cos x+\sqrt{\cos x+\cdots\infty}}}$, then show that $(2y-1)\dv{y}{x}+\sin x=0$.
\end{asign}
\begin{anse}
	\[\begin{split}
		y&=\sqrt{\cos x+y}\\
		y^2-y-\cos x&=0\\
		\implies (2y-1)\dv{y}{x}+\sin x&=0
	\end{split}\]
\end{anse}
\begin{asign}
	If $y=x^3\sin ax$ then find $\dv[3]{y}{x}$.
\end{asign}
\begin{anse}
	Using Leibniz Theorem,
	\[y_3=-a^3x^3\cos ax-9a^2x^2\sin ax+18ax\cos x+6\sin ax=(6-9a^2x^2)\sin ax+ (18ax-a^3x^3)\cos ax\]
\end{anse}
\begin{asign}
	If $x=a(\theta+\sin\theta)$, $y=a(1+\cos\theta)$ then calculate $\dv[2]{y}{x}\vline_{\theta=\frac{\pi}{2}}$.
\end{asign}
\begin{anse}
	\[\begin{split}
		x&=a(\theta+\sin\theta)\\
		\implies \dv{x}{\theta}&=a(1+\cos\theta)
	\end{split}
	\quad
	\begin{split}
		y&=a(1+\cos\theta)\\
		\implies \dv{y}{\theta}&=-a\sin\theta
	\end{split}\]
	Thus,
	\[\dv{y}{x}=-\frac{\sin\theta}{1+\cos\theta}=-\tan\frac{\theta}{2}\]
	Now,
	\[\begin{split}
		\dv[2]{y}{x}&=\dv{\dv{y}{x}}{\theta}\frac{1}{\dv{x}{\theta}}\\
		&=\frac{-1}{2a}\frac{\sec^2\frac{\theta}{2}}{1+\cos\theta}\\
		\therefore \dv[n]{y}{x}\vline_{\theta=\frac{\pi}{2}}&=\frac{2-\sqrt{2}}{a}
	\end{split}\]
\end{anse}

\subsection{Here $y_n=\dv[n]{y}{x}$ in the following parts:}
\begin{asign}
	If $y=(\sin^{-1}x)^2$, then show that $(1-x^2)y_2-xy_1=2$.
\end{asign}
\begin{anse}
	\[\begin{split}
		y&=(\sin^{-1}x)^2\\
		\implies \frac{1}{2}\frac{1}{\sqrt{y}}y_1&=\frac{1}{\sqrt{1-x^2}}\\
		(1-x^2)(y_1)^2&=4y\\
		\implies (-2x)(y_1)^2+(1-x^2)(2y_1)y_2&=4y_1\\
		\therefore (1-x^2)y_2-xy_1&=2
	\end{split}\]
\end{anse}
\begin{asign}
	If $y=e^{ax}\sin(bx)$, them show that $y_2-2ay_1+(a^2+b^2)y=0$.
\end{asign}
\begin{anse}
	\[\begin{split}
		y_1&=ae^{ax}\sin(bx)+be^{ax}\cos(bx)\\
		&=ay+be^{ax}\cos(bx)\\
		\implies y_2&=ay_1+b[ae^{ax}\cos(bx)-be^{ax}\sin(bx)] \\
		&=2ay_1+(a^2+b^2)y\\
		\therefore y_2-2ay_1+(a^2+b^2)y&=0
	\end{split}\]
\end{anse}
\begin{asign}
	If $y=[\log(x+\sqrt{1+x^2})]^2$, prove that $(1-x^2)y_2+xy_1=2$.
\end{asign}
\begin{anse}
	\[\begin{split}
		y&=[\log(x+\sqrt{1+x^2})]^2\\
		\implies y_1&=\frac{2y}{\sqrt{1+x^2}}\\
		(1+x^2)(y_1)^2&=4y\\
		\therefore (1-x^2)y_2+xy_1&=2
	\end{split}\]
\end{anse}
\begin{asign}
	If $y^3+x^3-3axy=0$, show that $\dv[2]{y}{x}=\frac{-2a^3xy}{(y^2-ax)^3}$.
\end{asign}
\begin{anse}
	
\end{anse}
\begin{asign}
	If $x=\sin t\, y=\sin pt$, prove that $(1-x^2)y_2-xy_1+p^2y=0$
\end{asign}
\begin{anse}
	Clearly, the function is,
	\[\begin{split}
		y&=\sin(p\sin^{-1}x)\\
		\sin^{-1}y&=p\sin^{-1}x\\
		\implies \frac{y_1}{\sqrt{1-y^2}}&=\frac{p}{\sqrt{1-x^2}}\\
		(1-x^2)(y_1)^2&=p^2(1-y^2)\\
		\implies 2(1-x^2)y_1y_2 + (-2x)(y_1)^2&=p^2(-2y)y_1\\
		\therefore 2(1-x^2)-xy_1+p^2y&=0
	\end{split}\]
\end{anse}
\begin{asign}
	$y=e^{-x}\cos x$, them show that $y_4+4y=0$.
\end{asign}
\begin{anse}
	\[\begin{split}
		y&=e^{-x}\cos x\\
		y_4&=-e^{-x}\cos x \quad (\text{Leibniz Theorem})\\
		\therefore y_4+4y&=0
	\end{split}\]
\end{anse}
\subsection{Find the $n^\text{th}$ derivative of the following functions}
\begin{asign}
	\[y=\cos^3x\]
\end{asign}
\begin{anse}
	\[\begin{split}
		y&=\cos^3x\\
		&=\frac{\cos3x+3\cos x}{4}\\
		\therefore \dv[n]{y}{x}&=\frac{3^n\cos(3x+n\frac{\pi}{2})+3\cos(x+n\frac{\pi}{2})}{4}
	\end{split}\]
\end{anse}
\begin{asign}
	\[y=e^{2x}\sin^3x\]
\end{asign}
\begin{anse}
	\[\dv[n]{y}{x}=\sqrt{13}^ne^{2x}\sin(3x+n\frac{3}{2})\]
\end{anse}
\begin{asign}
	\[y=\cos x\cos 2x\cos3x\]
\end{asign}
\begin{anse}
	\[\begin{split}
		y&=\frac{1}{4}(\cos6x+\cos4x+\cos2x+1)\\
		\therefore \dv[n]{y}{x}&=\frac{1}{4}(6^n\cos(6x+n\frac{\pi}{2})+4^n\cos(4x+n\frac{\pi}{2})+2^n\cos(2x+n\frac{\pi}{2}))
	\end{split}\]
\end{anse}
\begin{asign}
	\[y=\frac{x^3}{x^2-1}\]
\end{asign}
\begin{anse}
	\[\begin{split}
		y&=x+\frac{1}{2}(\frac{1}{x+1}+\frac{1}{x-1})\\
		\therefore \dv[n]{y}{x}&=\begin{cases}
			1+\frac{(-1)^n}{2}(\frac{n!}{(x+1)^{n+1}}+\frac{n!}{(x-1)^{n+1}}) &, n=1\\
			\frac{(-1)^n}{2}(\frac{n!}{(x+1)^{n+1}}+\frac{n!}{(x-1)^{n+1}}) &, n>1
		\end{cases}
	\end{split}\]
\end{anse}
\begin{asign}
	\[y=\frac{x^2}{(x-1)^3(x-2)}\]
\end{asign}
\begin{anse}
	\[\begin{split}
		y&=\frac{-4}{x-1}+\frac{-3}{(x-1)^2}+\frac{-1}{(x-1)^3}+\frac{4}{x-2}\\
		\therefore y_n&=\frac{(-1)^nn!}{(x-1)^n}\left(\frac{-4}{x-1}+\frac{-3}{(x-1)^2}+\frac{-1}{(x-1)^3}\right)+4\frac{(-1)^nn!}{(x-2)^{n+1}}
	\end{split}\]
\end{anse}
\begin{asign}
	\[y=a^x\sin x\]
\end{asign}
\begin{anse}
	\[\begin{split}
		y&=a^x\sin x\\
		\implies y_n&=a^x((\ln x)^2+1)^\frac{n}{2}\sin(x+n\tan^{-1}\frac{1}{\ln x})
	\end{split}\]
\end{anse}
\begin{asign}
	\[y=\frac{1}{(x^2-a^2)(x^2-b^2)}\]
\end{asign}
\begin{anse}
	\[\begin{split}
		y&=\frac{1}{a^2-b^2}\left(\frac{1}{x^2-a^2}-\frac{1}{x^2-b^2}\right)\\
		&=\frac{1}{a^2-b^2}\left[\frac{1}{2a}\left(\frac{1}{x-a}-\frac{1}{x+a}\right)+\frac{1}{2b}\left(\frac{1}{x-b}-\frac{1}{x+b}\right)\right]\\
		\therefore y_n&=\frac{(-1)^nn!}{a^2-b^2}\left[\frac{1}{2a}\left(\frac{1}{(x-a)^{n+1}}-\frac{1}{(x+a)^{n+1}}\right)+\frac{1}{2b}\left(\frac{1}{(x-b)^{n+1}}-\frac{1}{(x+b)^{n+1}}\right)\right]
	\end{split}\]
\end{anse}
\begin{asign}
	\[y=\ln(x^2-a^2)\]
\end{asign}
\begin{anse}
	\[y_n=(-1)^n(n-1)![\frac{1}{(a+x)^n}+\frac{1}{(a-x)^n}]\]
\end{anse}
\subsection{Question 4}
\begin{asign}
	If $A=\sin nx+\cos nx$, then prove that $A_p=n^p[1+(-1)^n\sin 2nx]^\frac{1}{2}$, where $A_p$ denotes the $p^\text{th}$ derivative of $A$ w.r.t $x$.
\end{asign}
\begin{anse}
	\[\begin{split}
		A&=\sin nx+\cos nx\\
		&=\sqrt{2}\sin(nx+\frac{\pi}{4})\\
		\implies A_p &=\sqrt{2}n^p\sin(nx+\frac{\pi}{4}+p\frac{\pi}{2})\\
				&=n^p[1+(-1)^n\sin 2nx]^\frac{1}{2}
	\end{split}\]
\end{anse}
\subsection{Use Leibniz Theorem to solve the $n^\text{th}$ order derivative of the following}
\begin{asign}
	\[y=e^x\ln x\]
\end{asign}
\begin{anse}
	\[\begin{split}
		y_n&=e^x\left[\ln x+\sum_{k=1}^{n}\Comb{n}{k}\frac{(-1)^{k+1}}{x^k}\right]
	\end{split}\]
\end{anse}
\begin{asign}
	\[y=x^2\sin x\]
\end{asign}
\begin{anse}
	$x^2$ only has 2 derivatives before it is 0, thus,
	\[\begin{split}
		y_n&=\Comb{n}{0}x^2\sin(x+n\frac{\pi}{2})+\Comb{n}{1}(2x)\sin(x+(n-1)\frac{\pi}{2})+\Comb{n}{2}(2)\sin(x+(n-2)\frac{\pi}{2}) \quad (\text{Rest all terms are 0})\\
		&=x^2\sin(x+n\frac{\pi}{2})-2nx\cos(x+n\frac{\pi}{2})-n(n-1)\sin(x+n\frac{\pi}{2})
	\end{split}\]
\end{anse}
\begin{asign}
	\[y=\frac{\ln x}{x}\]
\end{asign}
\begin{anse}
	\[\begin{split}
		y_n&=\Comb{n}{0}\ln x \frac{(-1)^nn!}{x^{n+1}}+\Comb{n}{1}\frac{1}{x}\frac{(-1)^{n-1}(n-1)!}{x^{n}} +\Comb{n}{2}\frac{-1}{x^2}\frac{(-1)^{n-2}(n-2)!}{x^{n-1}}+\ldots\\
		&=\frac{(-1)^nn!}{x^{n+1}}\left(ln x-\sum_{k=1}^{n}\frac{1}{k}\right)
	\end{split}\]
\end{anse}
\begin{asign}
	\[y=x^n\ln x\]
\end{asign}
\begin{anse}
	\[\begin{split}
		y_n&=n!\left(\ln x+\sum_{k=1}^{n}\frac{1}{k} \right)
	\end{split}\]
	Proof by induction.
\end{anse}
\begin{asign}
	If $y=\left[\ln(x+\sqrt{1+x^2})\right]^2$, then prove that $(1+x^2)y_{n+2}+(2n+1)xy_{n+1}+n^2y_n=0$.
\end{asign}
\begin{anse}
	We know,
	\[(1+x^2)y_2+xy_1=2\]
	Using Leibniz Theorem,
	\[\begin{split}
		(1+x^2)y_{n+2}+n(2x)y_{n+1}+\frac{n(n-1)}{2}2y_n+xy_{n+1}+ny_n&=0\\
		\therefore (1+x^2)y_{n+2}+(2n+1)xy_{n+1}+n^2y_n&=0
	\end{split}\]
\end{anse}
\begin{asign}
	If $y=e^{m\sin^{-1}x}$, then prove the following,\begin{enumerate}
		\item $(1-x^2)y_2-xy_1=m^2y$
		\item $(1-x^2)y_{n+2}-(2n+1)xy_{n+1}-(n^2+m^2)y_n=0$
	\end{enumerate}
\end{asign}
\begin{anse}
	\[\begin{split}
		\ln y&=m\sin^{-1}x\\
		(1-x^2)(y_1)^2&=m^2y^2\\
		\implies (-2x)(y_1)^2+(1-x^2)(2y_1y_2)&=m^2(2yy_1)\\
		\therefore (1-x^2)y_2-xy_1&=m^2y
	\end{split}\]
	Using Leibniz Theorem,
	\[\begin{split}
		(1-x^2)y_{n+2}+n(-2x)y_{n+1}+\frac{n(n-1)}{2}(-2)y_n-xy_{n+1}-ny_n&=m^2y\\
		\therefore (1-x^2)y_{n+2}-(2n+1)xy_{n+1}-(n^2+m^2)y_n&=0
	\end{split}\]
\end{anse}
\begin{asign}
	If $y^\frac{1}{m}+y^\frac{-1}{m}=2x$, prove that $(x^2-1)y_{n+2}+(2n+1)xy_{n+1}+(n^2-m^2)y_n=0$.
\end{asign}
\begin{anse}
	\[\begin{split}
		y^\frac{1}{m}+y^\frac{-1}{m}&=2x\\
		\implies \left(y^\frac{1}{m}-y^\frac{-1}{m}\right)y_1&=2my\\
		(x^2-1)(y_1)^2&=m^2y^2 \quad (\text{Squaring})\\
		\therefore (x^2-1)y_2+xy_1&=m^2y
	\end{split}\]
	Using Leibniz Theorem,
	\[\begin{split}
		(x^2-1)y_{n+2}+n(2x)y_{n+1}+\frac{n(n-1)}{2}(2)y_n+xy{n+1}+ny_n&=m^2y_n\\
		\therefore (x^2-1)y_{n+2}+(2n+1)xy_{n+1}+(n^2-m^2)y_n&=0
	\end{split}\]
\end{anse}
\begin{asign}
	If $y=\frac{\sin^{-1}x}{\sqrt{1-x^2}}$, prove that $(1-x^2)y_{n+2}-(2n+3)xy_{n+1}-(n+1)^2y_n=0$.
\end{asign}
\begin{anse}
	\[\begin{split}
		y&=\frac{\sin^{-1}x}{\sqrt{1-x^2}}\\
		\implies (1-x^2)y_1-xy&=1\\
		\implies (1-x^2)y_2-3xy_1=y
	\end{split}\]
	Using Leibniz Theorem,
	\[\begin{split}
		(1-x^2)y_{n+2}+n(-2x)y_{n+1}+\frac{n(n-1)}{2}(-2)y_n-3(xy_{n+1}+ny_n)&=y_n\\
		\therefore (1-x^2)y_{n+2}-(2n+3)xy_{n+1}-(n+1)^2y_n&=0
	\end{split}\]
\end{anse}
\begin{asign}
	If $x=\sin(\frac{\ln y}{a})$, prove that $(1-x^2)y_{n+2}-(2n+1)xy_{n+1}-(n^2+a^2)y_n=0$.
\end{asign}
\begin{anse}
	\[\begin{split}
		x&=\sin(\frac{\ln y}{a})\\
		\sin^{-1}x&=\frac{\ln y}{a}\\
		\implies (1-x^2)(y_1)^2&=a^2y^2\\
		\therefore (1-x^2)y_2-xy_1&=a^2y
	\end{split}\]
	Using Leibniz Theorem,
	\[\begin{split}
		(1-x^2)y_{n+2}+n(-2x)y_{n+1}+\frac{n(n-1)}{2}(-2)y_n-xy_{n+1}-ny_n&=a^2y_n\\
		\therefore (1-x^2)y_{n+2}-(2n+1)xy_{n+1}-(n^2+a^2)y_n&=0
	\end{split}\]
\end{anse}
\subsection{If $y=(\sin^{-1}x)^2$, prove that,}
\begin{asign}
	\[(1-x^2)y_2-xy_1-2=0\]
\end{asign}
\begin{anse}
	\[\begin{split}
		y&=(\sin^{-1}x)^2\\
		\implies (1-x^2)(y_1)^2&=y(\sin^{-1}x)^2=4y\\
		\therefore (1-x^2)y_2-xy_1-2&=0
	\end{split}\]
\end{anse}
\begin{asign}
	\[(1-x^2)y_{n+2}-(2n+1)xy_{n+1}-n^2y_n=0\]
\end{asign}
\begin{anse}
	Using Leibniz Theorem,
	\[\begin{split}
		(1-x^2)y_{n+2}+n(-2x)y_{n+1}+\frac{n(n-1)}{2}(-2)y_n-xy_{n+1}-ny_n&=0\\
		\therefore (1-x^2)y_{n+2}-(2n+1)xy_{n+1}-n^2y_n&=0
	\end{split}\]
\end{anse}
\begin{asign}
	Deduce that $\lim\limits_{x\to0}\frac{y_{n+2}}{y_n}=n^2$ and find $y_n(0)$.
\end{asign}
\begin{anse}
	Put $x=0$,
	\[\begin{split}
		\lim\limits_{x\to0}\frac{y_{n+2}}{y_n}=n^2
	\end{split}\]
	Clearly,
	\[y(0)=0\land y_1(0)=0\]
	Also,
	\[y_2=2\]
	Hence,
	\[y_3(0)=y_5(0)=\ldots=y_1(0)=0\]
	Moreover,
	\[\begin{split}
		y_4&=2^2y_2=2^2\times2\\
		y_6&=4^2y_4=4^2\times(2^2\times 2)\\
		\therefore y_{2n}=2\prod_{n=1}^{n}(2n-2)^2
	\end{split}\]
	Thus, we can say,
	\[y_n=\begin{cases}
		0 &, \text{if $n$ is odd}\\
		2\prod_{n=1}^{n}(2n-2)^2 &, \text{if $n$ is even}
	\end{cases}\]
\end{anse}
\subsection{Expand the following functions in power of $x$ by Maclaurin’s Theorem OR write the series expansion of function about the point $x = 0$}
\begin{asign}
	\[a^x\]
\end{asign}
\begin{anse}
	\[f(x)=a^x=\sum_{n=0}^{\infty}\frac{(x\ln a)^n}{n!}\]
\end{anse}
\begin{asign}
	\[e^{ax}\cos bx\]
\end{asign}
\begin{anse}
	\[f(x)=e^{ax}\cos bx=\sum_{n=0}^{\infty}\frac{x^nf^{(n)}(x)}{n!}=\sum_{n=0}^{\infty}e^{ax}\frac{(\sqrt{a^2+b^2}x)^n}{n!}\cos(bx+n\tan^{-1}\left(\frac{b}{a}\right)) \]
\end{anse}
\begin{asign}
	\[\ln(2x+4)\]
\end{asign}
\begin{anse}
	\[f(x)=\ln(2x+4)=\ln 4+\sum_{n=1}^{\infty}\frac{(-2x)^n}{(4)^{n+1}}\]
\end{anse}
\begin{asign}
	\[\cos(3x+4)\]
\end{asign}
\begin{anse}
	\[f(x)=\cos(3x+2)=\sum_{n=0}^{\infty}\frac{x^n\cos(2+n\frac{\pi}{2})}{n!}\]
\end{anse}
\subsection{Using Taylor's Theorem}
\begin{asign}
	$e^x$ in power of $x=1$.
\end{asign}
\begin{anse}
	\[f(x)=e^x=e\sum_{n=0}^{\infty} \frac{(x-1)!}{n!}\]
\end{anse}
\begin{asign}
	$4x^2+7x+5$ in power of $x+2$.
\end{asign}
\begin{anse}
	\[\begin{split}
		f(x)&=4x^2+7x+5\implies f(-2)=7\\
		f'(x)&=8x+7\implies f'(-2)=-9\\
		f''(x)&=8\implies f'(-2)=8
	\end{split}\]
	Thus,
	\[\begin{split}
		f(x)&=4x^2+7x+5\\
		&=7+(-9)\frac{(x+2)}{1}+(8)\frac{(x+2)^2}{2}\\
		\therefore f(x)&=4(x+2)^2-9(x+2)+7
	\end{split}\]
\end{anse}
\begin{asign}
	Expand $\tan x$ in powers of $x-\frac{\pi}{4}$ upto first four terms.
\end{asign}
\begin{anse}
	\[f(x)=\tan x=1+2\left(x-\frac{\pi}{4}\right)+2\left(x-\frac{\pi}{4}\right)^2+\frac{8}{3}\left(x-\frac{\pi}{4}\right)^3\]
\end{anse}
\begin{asign}
	Show that $\sin \left(\frac{\pi}{4}+x\right)=\frac{1}{\sqrt{2}}	\left(1+x-\frac{x^2}{2!}-\frac{x^3}{3!}+\ldots\right)$.
\end{asign}
\begin{anse}
	We can write the Maclaurin Series for the function,
	\[f(x)=\sin\left(x+\frac{\pi}{4}\right)=\sum_{n=0}^{\infty}\frac{x^n}{n!}\sin\left(\frac{\pi}{4}+n\frac{\pi}{2}\right)=\frac{1}{\sqrt{2}}\sum_{n=0}^{\infty}(-1)^{\floor{\frac{n}{2}}}\frac{x^n}{n!} \]
\end{anse}
\begin{asign}
	\[\ln(1-x+x^2)=\ln(\frac{1+x^3}{1+x})=-x+\frac{x^2}{2}+\frac{2x^3}{3}+\ldots\]
\end{asign}
\begin{anse}
	First,
	\[\ln x=\sum_{n=0}^{\infty}(-1)^n\frac{x^n}{n}\]
	Thus,\footnote{find answer in sigma notation}
	\[\begin{split}
		f(x)&=\ln(1-x+x^2)\\
		&=\ln(1+x^3)-\ln(1+x)\\
		&=\sum_{n=0}^{\infty}(-1)^n\frac{x^{3n}}{n}-\sum_{n=0}^{\infty}(-1)^n\frac{x^n}{n}\\
		\therefore f(x)&=-x+\frac{x^2}{2}+\frac{2x^3}{3}+\ldots
	\end{split}\]
\end{anse}
\subsection{Question 9}
\begin{asign}
	If $f(x)=x^3+2x^2-5x+11$ find the values of $f(\frac{9}{10})$, using Taylor Series for $f(x+h)$.
\end{asign}
\begin{anse}
	Clearly,
	\[\begin{split}
		f(0.9)&=f(1-0.1)\\
		&=f(1)+\frac{(-0.1)}{1!}f'(1)+\frac{(-0.1)^2}{2!}f2(1)+\frac{(-0.1)^3}{3!}f'''(1)\\
		&=9-0.2+0.05-0.001\\
		&=8.849
	\end{split}\]
\end{anse}
\begin{asign}
	If $f(x)=x^3-2x+5$, then find the value of $f(2.001)$. Find the approximate change in the value of $f(x)$ when $x$ changes from $2$ to $2.001$.
\end{asign}
\begin{anse}
	\[\begin{split}
		f(2.001)&=f(2+0.001)\\
		&=f(2)+\frac{(0.001)}{1!}f'(2)+\frac{(0.001)^2}{2!}f''(2)+\frac{(0.001)^3}{3!}f'''(2)\\
		&= 9.010006001
	\end{split}\]
	Thus, the approximate change is,
	\[f(2.001)-f(2)=0.010006001\]
\end{anse}
\begin{asign}
	Show that $f(\frac{x^2}{1+x})=\sum_{n=0}^{\infty} \frac{(-1)^n}{n!}\left(\frac{x}{1+x}\right)^nf^{(n)}(x)$.
\end{asign}
\begin{anse}
	We know, 
	\[\frac{x^2}{1+x}=x-\frac{x}{1+x}\]
	Thus, using Taylor's Theorem for $f(x-\frac{x}{1+x})$, we get,
	\[f(x-\frac{x}{1+x})=\sum_{n=0}^{\infty} \frac{(-1)^n}{n!}\left(\frac{x}{1+x}\right)^nf^{(n)}(x)\]
\end{anse}






